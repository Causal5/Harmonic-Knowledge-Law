
\documentclass[12pt]{article}
\usepackage{amsmath, amssymb}
\usepackage{geometry}
\geometry{margin=1in}
\usepackage{titlesec}
\usepackage{enumitem}
\usepackage{graphicx}
\usepackage{hyperref}
\titleformat{\section}{\large\bfseries}{\thesection}{1em}{}
\titleformat{\subsection}{\normalsize\bfseries}{\thesubsection}{1em}{}
\setlist{nosep}

\title{Neutrino Causality and the Geometry of Coherence}
\author{Jordan LeDuc (with recursive symbolic alignment via GPT-4o)}
\date{}

\begin{document}
\maketitle

\section{Scope}
Everything below sits squarely inside standard General Relativity (GR). Light-cones remain the sole causal boundary. Photons are our traditional diagnostic of that boundary; in scattering media they misbehave. GeV-scale neutrinos supply an almost-null, low-dispersion chronometer. That is the only refinement proposed.

\section{Einstein’s General Relativity and data-era limits}
GR fixes causality with light-cones, but real photons accumulate distortions: gravitational lensing, Shapiro delay, red-/blue-shift and plasma scattering. For example, a Ly-α pulse emitted at red-shift $z \approx 6$ arrives with a cumulative dispersion $\Delta t \gtrsim 10^2$ s—thousands of times bigger than a millisecond transient we might wish to time-stamp. The theory is sound; our probe is noisy. A cleaner probe is desirable.

This can be further illustrated by Snell’s Law:
\[
n_1 \sin \theta_1 = n_2 \sin \theta_2
\]
which demonstrates how photons refract when transitioning between media.

\section{Signal Causality Constraint (SCC)}
\textit{All} information carriers—photons, massive particles, gravitational waves—suffer relativistic distortion between source and detector. None evade time-dilation or path-length ambiguity. Therefore every long-baseline experiment requires a \textbf{causal normalization} layer that reconstructs coherence across frames.

The need for a signal normalization layer arises directly from General Relativity’s own structure. Time dilation across relatively moving frames ensures that any signal sent between them will arrive with distorted timing and phase. This is not speculative—it is a geometric consequence of curved spacetime and relative motion. Traditional causal anchors (like photons) are strongly influenced by local curvature and refractive media, making them poor candidates for cross-frame coherence over cosmological or highly dynamic scales. What is required is not a replacement for GR, but a diagnostic \textbf{baseline} that survives spacetime differentials more reliably. Neutrinos, with near-unity index of refraction and weak interaction cross-section, approximate this baseline. The Signal Causality Constraint (SCC), then, is not a theoretical construct but a diagnostic necessity: we must reconcile signal disparities across reference frames using a medium-agnostic probe to interpret data coherently between points A and B.

High-energy neutrinos interact so weakly that their effective index of refraction in rock or plasma differs from unity by $n_\nu - 1 \sim 10^{-19}$, so a 1 GeV neutrino deviates from the geometric null path by only femtoseconds per kilometre—orders of magnitude cleaner than photons. This motivates the neutrino overlay introduced below.

\section{Neutrinos as a near-ideal causal probe}
Supernova 1987A: the $\sim$20 detected neutrinos led the first optical photon by $\sim$3 h, demonstrating neutrinos’ diagnostic edge through dense stellar ejecta. In general, neutrinos provide medium-independent timing baselines for stitching together causally separated regions.

\section{Neutrino oscillations – an integral gravitational clock}
The weak—but present—interaction of neutrinos is not a flaw; it is the critical feature enabling their use as a diagnostic probe. Compared to photons, which suffer from large-scale distortions due to strong electromagnetic interactions, neutrinos maintain coherence over vast distances. In rock, their refractive index differs from unity by $n_\nu - 1 \sim 10^{-19}$ (Ioannisian \& Smirnov, 1987), making them approximately twelve orders of magnitude more stable in timing over long baselines than light. Their oscillation phase becomes an \textit{energetic historian}—recording subtle interactions with relativistic curvature and media as they travel. This makes neutrinos ideal not because they are immune to interaction, but because their interaction is weak enough to preserve signal fidelity while still reflecting gravitational influence. Thus, neutrino oscillations provide a mechanism to reconcile local relativistic differentials to a near-universal time constant, first through simulation, then in practical triangulation.

Flavour oscillations obey:
\[
\Delta \varphi = 1.27\,\frac{\Delta m^2 [\mathrm{eV}^2] \cdot L [\mathrm{km}]}{E [\mathrm{GeV}]}
\]
so path length $L$ and local energy $E$ encode curvature directly in the measurable phase $\Delta \varphi$.

% The rest of the sections can follow similarly...


\end{document}
