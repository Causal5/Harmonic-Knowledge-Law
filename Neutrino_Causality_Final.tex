
\documentclass[12pt]{article}
\usepackage{amsmath, amssymb}
\usepackage{geometry}
\geometry{margin=1in}
\usepackage{titlesec}
\usepackage{enumitem}
\usepackage{graphicx}
\usepackage{hyperref}
\titleformat{\section}{\large\bfseries}{\thesection}{1em}{}
\titleformat{\subsection}{\normalsize\bfseries}{\thesubsection}{1em}{}
\setlist{nosep}

\title{Neutrino Causality and the Geometry of Coherence}
\author{Jordan LeDuc (with recursive symbolic alignment via GPT-4o)}
\date{}

\begin{document}
\maketitle

\section{Scope}
Everything below sits squarely inside standard General Relativity (GR). Light-cones remain the sole causal boundary. Photons are our traditional diagnostic of that boundary; in scattering media they misbehave. GeV-scale neutrinos supply an almost-null, low-dispersion chronometer. That is the only refinement proposed.

\section{Einstein’s General Relativity and data-era limits}
GR fixes causality with light-cones, but real photons accumulate distortions: gravitational lensing, Shapiro delay, red-/blue-shift and plasma scattering. For example, a Ly-α pulse emitted at red-shift $z \approx 6$ arrives with a cumulative dispersion $\Delta t \gtrsim 10^2$ s—thousands of times bigger than a millisecond transient we might wish to time-stamp. The theory is sound; our probe is noisy. A cleaner probe is desirable.

This can be further illustrated by Snell’s Law:
\[
n_1 \sin \theta_1 = n_2 \sin \theta_2
\]
which demonstrates how photons refract when transitioning between media.

\section{Signal Causality Constraint (SCC)}
\textit{All} information carriers—photons, massive particles, gravitational waves—suffer relativistic distortion between source and detector. None evade time-dilation or path-length ambiguity. Therefore every long-baseline experiment requires a \textbf{causal normalization} layer that reconstructs coherence across frames.

The need for a signal normalization layer arises directly from General Relativity’s own structure. Time dilation across relatively moving frames ensures that any signal sent between them will arrive with distorted timing and phase. This is not speculative—it is a geometric consequence of curved spacetime and relative motion. Traditional causal anchors (like photons) are strongly influenced by local curvature and refractive media, making them poor candidates for cross-frame coherence over cosmological or highly dynamic scales. What is required is not a replacement for GR, but a diagnostic \textbf{baseline} that survives spacetime differentials more reliably. Neutrinos, with near-unity index of refraction and weak interaction cross-section, approximate this baseline. The Signal Causality Constraint (SCC), then, is not a theoretical construct but a diagnostic necessity: we must reconcile signal disparities across reference frames using a medium-agnostic probe to interpret data coherently between points A and B.

High-energy neutrinos interact so weakly that their effective index of refraction in rock or plasma differs from unity by $n_\nu - 1 \sim 10^{-19}$, so a 1 GeV neutrino deviates from the geometric null path by only femtoseconds per kilometre—orders of magnitude cleaner than photons.

\section{Neutrinos as a near-ideal causal probe}
Supernova 1987A: the $\sim$20 detected neutrinos led the first optical photon by $\sim$3 h, demonstrating neutrinos’ diagnostic edge through dense stellar ejecta. In general, neutrinos provide medium-independent timing baselines for stitching together causally separated regions.

\section{Neutrino oscillations – an integral gravitational clock}
The weak—but present—interaction of neutrinos is not a flaw; it is the critical feature enabling their use as a diagnostic probe. Compared to photons, which suffer from large-scale distortions due to strong electromagnetic interactions, neutrinos maintain coherence over vast distances. In rock, their refractive index differs from unity by $n_\nu - 1 \sim 10^{-19}$ (Ioannisian \& Smirnov, 1987), making them approximately twelve orders of magnitude more stable in timing over long baselines than light. Their oscillation phase becomes an \textit{energetic historian}—recording subtle interactions with relativistic curvature and media as they travel. This makes neutrinos ideal not because they are immune to interaction, but because their interaction is weak enough to preserve signal fidelity while still reflecting gravitational influence.

Flavour oscillations obey:
\[
\Delta \varphi = 1.27\,\frac{\Delta m^2 [\mathrm{eV}^2] \cdot L [\mathrm{km}]}{E [\mathrm{GeV}]}
\]
so path length $L$ and local energy $E$ encode curvature directly in the measurable phase $\Delta \varphi$.

\section{From photon-based differentials to neutrino-based integrals}
Photons excel at measuring \textit{gradients} of curvature (lensing angles, red-shift), but accumulate dispersion over large baselines. Neutrino phase instead furnishes an \textbf{integral} constraint with minimal error, tightening global causal bookkeeping.

\section{Causal triangulation via neutrino geometry}
Two phased beams from sites A and B that meet at C form a 2-D causal slice. Knowing $\Delta \varphi$ on each leg fixes the baseline length to $< 10^{-18}$ relative error; photons confined to the same triangle then map local curvature.

\section{Neutrinos = global chronometers, Photons = local tracers}
\textbf{Global probe.} With $n_\nu - 1 \sim 10^{-19}$ the time-of-flight offset is $\lesssim 10$ fs/km, so a phase-locked GeV beam gives a medium-agnostic reference across opaque or multiply-lensed regions.

\textbf{Local probe.} Photons remain unmatched for microlensing or Shapiro-delay measurements, but in dust, plasma or glass the same interactions that ease detection ruin long-baseline timing. The two probes are therefore complementary, not competitive.

\section{Philosophical note — extending, not replacing, GR}
Einstein lacked neutrino astronomy. Adding a neutrino chronometer supplies a \textbf{practical} medium-agnostic clock he could not test; the field equations and light-cone structure remain unchanged.

\section{Links to Geometric Intelligence \& Causal Ethics}
A robust, low-entropy causal scaffold aligns with Geometric Intelligence’s aim of minimal-action information flow and with Causal Ethics’ requirement that interventions respect global coherence.

\section{Future work}
\begin{itemize}
\item Numerically simulate neutrino-phase triangulation.
\item Test phase-shift sensitivity in IceCube-upgrade baselines.
\item Couple the overlay to SLP/GILN knowledge graphs.
\end{itemize}

\section{Interstellar relativistic communication example}
Between Earth and an Alcubierre-metric craft “moving” at 1.3 $c$ relative to origin frame, photon channels endure frame-dependent lags and severe lensing. A neutrino phase-slice anchored at both ends provides the constant-phase surface:
\begin{itemize}
\item Compensate time-dilation via $\Delta \varphi$ interpolation.
\item Maintain semantic integrity of packet headers.
\item Treat photon geodesics as local curvature corrections only.
\end{itemize}

\section{Technical appendix — neutrino planar overlay}
Construct a 2-surface $\Sigma$ defined by constant neutrino phase between events A and B. Solve only the local GR correction near each end:
\[
S_{AB} = \int_A^B \frac{ds_\nu}{E(s_\nu)} + \sum_i \Delta t_i(g_{\mu\nu}^{(i)})
\]
where the first term is evaluated on $\Sigma$ (global chronometer) and each $\Delta t_i$ is a small local correction derived from the spacetime metric.

\section*{Glossary}
\begin{itemize}
\item \textbf{Alcubierre metric} – A speculative curved-spacetime solution permitting super-luminal coordinate velocities without local $v > c$.
\item \textbf{Index of refraction (neutrino)} – Deviation from unity $n_\nu - 1 \sim 10^{-19}$ in rock.
\item \textbf{Signal Causality Constraint (SCC)} – A proposed diagnostic model ensuring coherence across relativistically distorted signal paths.
\item \textbf{General Relativity (GR)} – Einstein’s theory describing gravity as the curvature of spacetime caused by mass and energy.
\item \textbf{Photon} – A quantum of light; traditionally used as the causal reference in relativity.
\item \textbf{Neutrino} – A nearly massless, weakly interacting particle capable of traversing vast cosmic distances with minimal distortion.
\item \textbf{Neutrino Oscillation} – The quantum phenomenon by which neutrinos change flavor as they travel, sensitive to path length and energy.
\item \textbf{Causal Triangulation} – Geometric method using neutrino signals to stabilize reference frames in spacetime.
\item \textbf{Paraboloid Geometry} – Shape representing local curvature in GR.
\item \textbf{Spacetime Distortion} – The warping of spacetime due to gravitational fields.
\item \textbf{Time Dilation} – The slowing of time in strong gravitational fields or high-speed frames.
\item \textbf{Geometric Intelligence} – A framework linking intelligence propagation to stable geometric structures.
\item \textbf{Causal Ethics} – A framework that aligns ethics with causal coherence across systems.
\item \textbf{Law of Cosines} – A trigonometric law used for causal triangulation.
\item \textbf{Planar Overlay} – A 2D reference constructed from neutrino behavior.
\item \textbf{Phase Clock} – Oscillatory phase of neutrinos used as a distributed timing mechanism.
\item \textbf{Causal Normalization} – Reconstruction of signal coherence between time-dilated reference frames.
\item \textbf{Metric Perturbation} – Local deformation of the metric tensor used as a correction to global integrals.
\item \textbf{Null Geodesic} – Lightlike path through curved spacetime.
\item \textbf{Shapiro Delay} – Time delay due to gravitational influence near massive bodies.
\item \textbf{Phase-Locked Beam} – A neutrino stream emitted with fixed phase relations.
\end{itemize}

\end{document}
